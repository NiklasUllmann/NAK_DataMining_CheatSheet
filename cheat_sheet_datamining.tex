\documentclass[a4paper]{article}

\usepackage{pdflscape}
\usepackage{multicol}
\usepackage{blindtext}
\usepackage{color}
\usepackage{enumitem}

\usepackage[left=10mm, right=10mm, top=10mm, bottom=10mm]{geometry}

\usepackage{titlesec}

\usepackage[utf8]{inputenc}
\usepackage{fourier} 
\usepackage{array}
\usepackage{makecell}

\renewcommand\theadalign{bc}
\renewcommand\theadfont{\bfseries}
\renewcommand\theadgape{\Gape[4pt]}
\renewcommand\cellgape{\Gape[4pt]}

\titlespacing\section{0pt}{5pt plus 4pt minus 2pt}{0pt plus 2pt minus 2pt}
\titlespacing\subsection{0pt}{5pt plus 4pt minus 2pt}{0pt plus 2pt minus 2pt}
\titlespacing\subsubsection{0pt}{5pt plus 4pt minus 2pt}{0pt plus 2pt minus 2pt}



\setlength{\columnseprule}{0.5pt}
\def\columnseprulecolor{\color{black}}

\pagenumbering{gobble}

\title{DataMining Cheat Sheet}
\author{Niklas Ullmann}
\date{Summer 2022}


\begin{document}
\begin{landscape}
    \thispagestyle{empty}

    \begin{multicols}{3}
        
    \section{Grundlagen}
        \begin{itemize}[noitemsep,nolistsep,leftmargin=*]
            \item Qualitative Attribute: 
                \begin{itemize}
                    \item Variieren nach Beschaffenheit
                \end{itemize}
            \item Quantitative Attribute:
                \begin{itemize}
                    \item Variieren nach Wert/Zahlen
                \end{itemize}
        \end{itemize}
        \begin{itemize}[noitemsep,nolistsep,leftmargin=*]
            \item Diskrete Attribute: 
                \begin{itemize}
                    \item abgestufte Werte
                \end{itemize}
            \item Stetige Attribute:
                \begin{itemize}
                    \item können im Intervall jeden reellen Wert annehmen
                \end{itemize}
        \end{itemize}
            \subsection{Skalenniveaus}
            \begin{itemize}[noitemsep,nolistsep,leftmargin=*]
                \item Nominal
                \begin{itemize}
                    \item nur Gleichheit oder Andersartigkeit feststellbar (keine Bewertung)
                    \item stets qualitativ
                \end{itemize}
                \item Ordinal
                \begin{itemize}
                    \item natürliche oder festzulegende Rangfolge
                \end{itemize}
                \item Kardinal/Metrisch
                \begin{itemize}
                    \item numerischer Art 
                    \item Ausprägung und Unterschied sind messbar
                    \item verhältnisskaliert (Absoluter Nullpunkt vorhanden; (Doppelt so viel.))
                    \item intervallskaliert (Kein Nullpunkt, nur Differenzen)
                \end{itemize}
            \end{itemize}
    \subsection{asym. Attribute}

    - Rauschen Artefakte, Ausreißer

    \subsection{Datenvorverarbeitung}

    \subsection{Ähnlichkeits- und Distanzmaße}
            \subsubsection{Ähnlichkeit}

            \subsubsection{Distanz}

        
        - Mutual Information
        - Ähnlichkeit <-> Distanz (Umrechnung)

    \section{Klassifikation}

    

    \section{Clustering}
    
    
    \end{multicols}

    \newpage
    \section*{Übungsaufgaben und Musterlösungen}
\end{landscape}
\end{document}
