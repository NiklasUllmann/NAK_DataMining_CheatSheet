\documentclass[a4paper]{article}

\usepackage{pdflscape}
\usepackage{multicol}
\usepackage{blindtext}
\usepackage{color}
\usepackage{enumitem}

\usepackage[left=10mm, right=10mm, top=10mm, bottom=10mm]{geometry}

\usepackage{titlesec}

\usepackage[utf8]{inputenc}
\usepackage{fourier} 
\usepackage{array}
\usepackage{makecell}
\usepackage{amsmath}


\renewcommand\theadalign{bc}
\renewcommand\theadfont{\bfseries}
\renewcommand\theadgape{\Gape[4pt]}
\renewcommand\cellgape{\Gape[4pt]}

\titlespacing\section{0pt}{5pt plus 4pt minus 2pt}{0pt plus 2pt minus 2pt}
\titlespacing\subsection{0pt}{5pt plus 4pt minus 2pt}{0pt plus 2pt minus 2pt}
\titlespacing\subsubsection{0pt}{5pt plus 4pt minus 2pt}{0pt plus 2pt minus 2pt}



\setlength{\columnseprule}{0.5pt}
\def\columnseprulecolor{\color{black}}

\pagenumbering{gobble}

\title{DataMining Cheat Sheet}
\author{Niklas Ullmann}
\date{Summer 2022}


\begin{document}
\begin{landscape}
    \thispagestyle{empty}

    \begin{multicols}{3}
        
    \section{Grundlagen}
        \begin{itemize}[noitemsep,nolistsep,leftmargin=*]
            \item Qualitative Attribute: 
                \begin{itemize}
                    \item Variieren nach Beschaffenheit
                \end{itemize}
            \item Quantitative Attribute:
                \begin{itemize}
                    \item Variieren nach Wert/Zahlen
                \end{itemize}
        \end{itemize}
        \begin{itemize}[noitemsep,nolistsep,leftmargin=*]
            \item Diskrete Attribute: 
                \begin{itemize}
                    \item abgestufte Werte
                \end{itemize}
            \item Stetige Attribute:
                \begin{itemize}
                    \item können im Intervall jeden reellen Wert annehmen
                \end{itemize}
        \end{itemize}
            \subsection{Skalenniveaus}
            \begin{itemize}[noitemsep,nolistsep,leftmargin=*]
                \item Nominal
                \begin{itemize}
                    \item nur Gleichheit oder Andersartigkeit feststellbar (keine Bewertung)
                    \item stets qualitativ
                \end{itemize}
                \item Ordinal
                \begin{itemize}
                    \item natürliche oder festzulegende Rangfolge
                \end{itemize}
                \item Kardinal/Metrisch
                \begin{itemize}
                    \item numerischer Art 
                    \item Ausprägung und Unterschied sind messbar
                    \item verhältnisskaliert (Absoluter Nullpunkt vorhanden; (Doppelt so viel.))
                    \item intervallskaliert (Kein Nullpunkt, nur Differenzen)
                \end{itemize}
            \end{itemize}
    \subsection{Sym. vs asym. Attribute}
        \begin{itemize}[noitemsep,nolistsep,leftmargin=*]
            \item Das symmetrische binäre Attribut ist ein Attribut, bei dem jeder Wert gleichwertig ist (w/m)
            \item Asymmetrisch ist ein Attribut, bei dem die beiden Ausprägungen nicht gleichwertig sind (Testergebnisse oder Vergleich von Umfragen)
        \end{itemize}
    - Rauschen Artefakte, Ausreißer

    \subsection{Datenvorverarbeitung}

    \begin{itemize}[noitemsep,nolistsep,leftmargin=*]
        \item Aggregation
        \item Sampling
        \item Diskretisierung / Binarisierung
        \item Transformation
        \item Dimensionsreduktion
        \item Feature Subset Selection
        \item Feature Creation
    \end{itemize}

    \subsection{Ähnlichkeits- und Distanzmaße}
            \subsubsection{Ähnlichkeit}

            Eigenschaften:
            \begin{itemize}[noitemsep,nolistsep,leftmargin=*]
                \item $s(x,y) 0 <= s <= 1$
                \item $s(x,y) = 0$, wenn $x = y$
                \item Symmetry: $s(x,y) = s(y,x)$
            \end{itemize}

            Simple Matching Coefficient (SMC):
            \begin{itemize}
                \item $ SMC = \dfrac{f_{00}+f_{11}}{f_{01}+ f_{10}+ f_{00}+ f_{11}} $
            \end{itemize}

            Jaccard Coefficient:

            Extended Jaccard Coefficient (Tanimoto)

            Cosine Similarity:

            Correlation:


            \subsubsection{Distanz (Minkowski)}

            Eigenschaften:
            \begin{itemize}[noitemsep,nolistsep,leftmargin=*]
                \item Positivity (d(x,y) >= 0, d(x,y) = 0, wenn x = y)
                \item Symmetry (d(x,y)= d(y,x))
                \item Triangle Inequality (d(x,z) <= d(x,y) + d(y,z))
            \end{itemize}
            
            $$ d(x,y) =  \sqrt[r]{\sum^{n}_{k=1} | x_k - y_k |^r} $$

            \begin{center}
                
                \begin{tabular}{|l|l|l|}
                \hline
                Name      & r   & Anwendung                                                                                            \\ \hline
                Hamming   & 1   & Bin.Vekt. \\ \hline
                CityBlock & 1   &  nur gerade                                                                                                    \\ \hline
                Euclid    & 2   &  schräg                                                                                                    \\ \hline
                Supremum  & $\infty$ &  nur größte Dist.                                                                                                 \\ \hline
                \end{tabular}
            \end{center}

            \

            \subsubsection{Weiteres}

            Verhalten für Multiplikation und Addition:
            \begin{center}
            \begin{tabular}{|l|l|l|l|}
                \hline
                Property                              & Cosine & Correlation & Minkowski \\ \hline
                Invariant to multiplication & Yes    & Yes         & No        \\ \hline
                Invariant to addition   & No     & Yes         & No        \\ \hline
                \end{tabular}
            \end{center}
        
            Mutual Information:
            \begin{itemize}[noitemsep,nolistsep,leftmargin=*]
                \item Ähnlich wie Correlation, aber für nicht linearen Zusammenhang
                \item $0 =$ kein Zusammenhang, $1 =$ starker Zusammenhang
            \end{itemize}
            
            Umrechnung Ähnlichkeit $<->$ Distanz

    \section{Klassifikation}

    

    \section{Clustering}
    
    
    \end{multicols}

    \newpage
    \section*{Übungsaufgaben und Musterlösungen}
\end{landscape}
\end{document}
